%==== KNITR ===================================================================%



%==== START ===================================================================%

\documentclass{report}\usepackage[]{graphicx}\usepackage[]{xcolor}
% maxwidth is the original width if it is less than linewidth
% otherwise use linewidth (to make sure the graphics do not exceed the margin)
\makeatletter
\def\maxwidth{ %
  \ifdim\Gin@nat@width>\linewidth
    \linewidth
  \else
    \Gin@nat@width
  \fi
}
\makeatother

\definecolor{fgcolor}{rgb}{0.345, 0.345, 0.345}
\newcommand{\hlnum}[1]{\textcolor[rgb]{0.686,0.059,0.569}{#1}}%
\newcommand{\hlstr}[1]{\textcolor[rgb]{0.192,0.494,0.8}{#1}}%
\newcommand{\hlcom}[1]{\textcolor[rgb]{0.678,0.584,0.686}{\textit{#1}}}%
\newcommand{\hlopt}[1]{\textcolor[rgb]{0,0,0}{#1}}%
\newcommand{\hlstd}[1]{\textcolor[rgb]{0.345,0.345,0.345}{#1}}%
\newcommand{\hlkwa}[1]{\textcolor[rgb]{0.161,0.373,0.58}{\textbf{#1}}}%
\newcommand{\hlkwb}[1]{\textcolor[rgb]{0.69,0.353,0.396}{#1}}%
\newcommand{\hlkwc}[1]{\textcolor[rgb]{0.333,0.667,0.333}{#1}}%
\newcommand{\hlkwd}[1]{\textcolor[rgb]{0.737,0.353,0.396}{\textbf{#1}}}%
\let\hlipl\hlkwb

\usepackage{framed}
\makeatletter
\newenvironment{kframe}{%
 \def\at@end@of@kframe{}%
 \ifinner\ifhmode%
  \def\at@end@of@kframe{\end{minipage}}%
  \begin{minipage}{\columnwidth}%
 \fi\fi%
 \def\FrameCommand##1{\hskip\@totalleftmargin \hskip-\fboxsep
 \colorbox{shadecolor}{##1}\hskip-\fboxsep
     % There is no \\@totalrightmargin, so:
     \hskip-\linewidth \hskip-\@totalleftmargin \hskip\columnwidth}%
 \MakeFramed {\advance\hsize-\width
   \@totalleftmargin\z@ \linewidth\hsize
   \@setminipage}}%
 {\par\unskip\endMakeFramed%
 \at@end@of@kframe}
\makeatother

\definecolor{shadecolor}{rgb}{.97, .97, .97}
\definecolor{messagecolor}{rgb}{0, 0, 0}
\definecolor{warningcolor}{rgb}{1, 0, 1}
\definecolor{errorcolor}{rgb}{1, 0, 0}
\newenvironment{knitrout}{}{} % an empty environment to be redefined in TeX

\usepackage{alltt}

\usepackage[left=2cm, right=2cm, top=1cm, bottom=2cm]{geometry}

% Font.


% Main packages.
\usepackage[utf8]{inputenc}
\usepackage[T1]{fontenc}
\usepackage{amsmath}
\usepackage{graphicx}
\usepackage{hyperref}
\usepackage{booktabs} 
\usepackage{rotating} 
\usepackage{lmodern}

% Required for Table.


%%

\title{OeNB Industry Lab - Documentation}
\author{Tristan Leiter}
\date{\today}

%==== DOCUMENT START ==========================================================%

\IfFileExists{upquote.sty}{\usepackage{upquote}}{}
\begin{document}

\maketitle

%==== ABSTRACT ================================================================%

% \begin{abstract}
% This report demonstrates the integration of R code and its output within a LaTeX document using Sweave. It covers the basic structure of a report, including a summary, chapters with subchapters, and a bibliography.
% \end{abstract}

%==== Table of content ========================================================%

\tableofcontents
\newpage

%==== Introduction ============================================================%

\chapter{Einleitung}

\section{Einleitung}




%==== Chapter 1: General ======================================================%

\chapter{Exploratory Data Analysis}

\section{Overview}

\subsection{Description of the dataset}

\subsection{Description of the dataset}

\section{Data Splitting with Imbalanced Data}

\subsection{Class Imbalance}

In this credit risk dataset, the target variable, y, is inherently imbalanced. The number of non-defaults (0) significantly outnumbers the number of defaults (1). This is a common and expected characteristic of credit risk data.

This imbalance poses a significant challenge for model development. If we were to use a simple random split to create our training, validation, and test sets, we would face a high risk of creating unrepresentative samples. For example, a small test set could, purely by chance, end up with a much higher or lower percentage of defaults than the original dataset—or, in the worst case, zero defaults.

\subsection{The Solution: Stratified Sampling}

To prevent this, we employ stratified sampling. This is a technique that ensures the original class distribution of the target variable is preserved in each of the new data splits.

Here is the reasoning for its use:

Guarantees Representation: Stratification forces the splits to maintain the original ratio of defaults to non-defaults. If 0.0865\% of the original dataset are defaults, the training set, validation set, and test set will all contain approximately 0.0865\% defaults.

Enables Reliable Evaluation: When the test set is representative, the performance metrics we calculate (like accuracy, precision, recall, and F1-score) are meaningful. Evaluating a model on a test set with a skewed default rate would give us a misleading and over-optimistic (or pessimistic) score.

Promotes Model Generalization: By training the model on a set that accurately reflects the real-world data distribution, we help it learn the patterns of both the majority (non-default) and minority (default) classes, leading to a more robust and generalizable model.

In summary, using stratified sampling on the y variable is a critical step to ensure our model is trained and evaluated on a reliable, representative foundation.

%==== Chapter 2: Defining the loss-function ===================================%

\chapter{Loss-function}

\section{Area under the curve (AuC)}

The **Area Under the Curve (AUC)** is a single, aggregate metric that evaluates the performance of a binary classification model across all possible classification thresholds.

It is the area under the **ROC (Receiver Operating Characteristic) curve** , which plots the model's \textbf{Sensitivity} (True Positive Rate) against its \textbf{Specificity} (True Negative Rate) at every conceivable threshold.

\begin{itemize}
    \item An \textbf{AUC of 1.0} represents a perfect model that can distinguish between positive and negative classes with 100\% accuracy.
    \item An \textbf{AUC of 0.5} represents a model with no discriminatory power, equivalent to a random guess (as shown by the diagonal line in the plot).
    \item A good model will have an AUC well above 0.5.
\end{itemize}

\textbf{Relevance to Credit Risk:}
The AUC is extremely valuable because it measures the model's ability to \textit{rank} clients by risk. It answers the question: "If I pick a random defaulting firm and a random non-defaulting firm, what is the probability that my model gives a higher risk score to the defaulting firm?"

A high AUC (e.g., > 0.75) means the model is effective at separating the "bad" clients from the "good" ones. This is crucial before deciding on a specific \textit{business-level threshold} (like "what probability score triggers a loan rejection?"). It tells us the model itself is fundamentally sound.

% \begin{figure}[htbp]
%   \centering
%   \framebox{\parbox{0.8\textwidth}{\centering
%     \vspace{5cm}
%     \textbf{Your ROC Curve Plot} \\
%     \small\textit{This plot shows the trade-off between
%     Sensitivity and Specificity.}
%     \vspace{5cm}
%   }}
%   \caption{ROC Curve for the glmnet (Lasso) model.}
%   \label{fig:roc_curve}
% \end{figure}

\section{Recall}

\textbf{Recall}, also known as \textbf{Sensitivity} or the \textbf{True Positive Rate (TPR)}, is a metric that measures a model's ability to identify all relevant instances of a class.

It is calculated using the following formula, based on the outputs of a \textbf{confusion matrix} at a specific threshold:

\[
\text{Recall} = \frac{\text{True Positives (TP)}}{\text{True Positives (TP)} + \text{False Negatives (FN)}}
\]

\begin{itemize}
    \item \textbf{True Positives (TP):} The number of firms that \textit{actually defaulted} and were \textit{correctly flagged} as defaults by the model.
    \item \textbf{False Negatives (FN):} The number of firms that \textit{actually defaulted} but were \textit{incorrectly flagged} as non-defaults. This is the most costly error in credit risk.
\end{itemize}

\textbf{Relevance to Credit Risk:}
Recall answers the single most important question for a risk-averse institution: \textbf{"Of all the firms that actually defaulted, what percentage did we successfully catch?"}

In credit risk, the cost of a \textbf{False Negative} (missing a default) is extremely high, as it results in a direct financial loss. The cost of a \textbf{False Positive} (flagging a healthy firm as risky) is much lower---it might lead to a manual review or a rejected loan application (an opportunity cost).

Therefore, banks often prioritize high Recall. They are willing to accept a higher number of false alarms to ensure they minimize the number of defaults that slip through undetected. Setting a threshold to achieve a specific Recall (e.g., 95\%) is a common, risk-averse strategy.
%==== Chapter 3: Generalized linear models (GLMs) =============================%

\chapter{Generalized linear models (GLM)}

\section{Overview}

%==== Chapter 4: Regularized GLMS =============================================%


\chapter{Regularized GLMs}

\section{Description of the algorithm}

The regularized estimator is obtained by solving the following optimization problem:

\[
\hat{\beta}^{pen} = \arg \min_{\beta} \left\{-\frac{1}{N}l(\beta, \phi | \mathbf{y}, \mathbf{X}) + \lambda P(\beta)\right\}
\]

This objective function consists of two parts. In our logistic regression model, we predict the probability of default for each firm. We assume that the outcome (Y) is a Bernoulli random variable (a special case of the Binomial distribution). Then we model the probability of this specific firm defaulting given their features. This is reflected in the loss function:

\begin{enumerate}
    \item \textbf{The Loss Function: $\mathbf{-\frac{1}{N}l(\beta, \phi | \mathbf{y}, \mathbf{X})}$}
    
    This term measures how well the model fits the data.
    \begin{itemize}
        \item $l(\beta, \phi | \mathbf{y}, \mathbf{X})$ is the **log-likelihood function**. Since we assume a Bernoulli/Binomial outcome, this is the log-likelihood of the binomial distribution.
        \item We want to find parameters ($\beta$) that \textit{maximize} the likelihood of observing our data.
        \item Maximizing the log-likelihood is equivalent to \textit{minimizing} the **negative log-likelihood** (which is why the negative sign is there).
        \item The $\frac{1}{N}$ term scales it by the number of observations $N$ to get the **average negative log-likelihood**. This is also known as the \textbf{log-loss} or \textbf{cross-entropy loss}.
    \end{itemize}

    \item \textbf{The Penalty Term: $\mathbf{\lambda P(\beta)}$}
    
    This term, where $ \lambda \ge 0 $, penalizes the size of the coefficients to prevent overfitting. The \texttt{alpha} parameter controls the type of penalty:
    \begin{itemize}
        \item \texttt{alpha = 1} corresponds to the **Lasso** estimator (L1 penalty).
        \item \texttt{alpha = 0} corresponds to the **Ridge** estimator (L2 penalty).
    \end{itemize}
\end{enumerate}

The optimal regularization parameter, $ \lambda $, is found by $k$-fold cross-validation. For example, for $k=10$, the algorithm splits the training data into 10 equal-sized folds. It then iterates 10 times:

\begin{itemize}
    \item In each iteration, it trains the model on 9 folds and evaluates it on the 1 held-out fold (e.g., train on folds 2-10, test on fold 1).
    \item This entire process is repeated for a sequence of different $ \lambda $ values.
\end{itemize}

Finally, the algorithm calculates the average error (e.g., AUC or deviance) across all 10 folds for each $ \lambda $ value. The $ \lambda $ that produces the lowest average error is chosen as the optimal value.

In the \texttt{glmnet} package, this is implemented by:

\begin{verbatim}
cv_model_logit <- cv.glmnet(x_train, 
                            y_train, 
                            family = "binomial", 
                            alpha = 1)
\end{verbatim}

The optimal $ \lambda $ value is stored in \verb|cv_model_logit$lambda.min|, which is then used by the \texttt{predict()} function.

\subsection{Description of the algorithm}


%==== Anhang ==================================================================%

\appendix 

\chapter{Overview}

%==== END =====================================================================%

\end{document}
