%==== KNITR ===================================================================%



%==== START ===================================================================%

\documentclass{report}\usepackage[]{graphicx}\usepackage[]{xcolor}
% maxwidth is the original width if it is less than linewidth
% otherwise use linewidth (to make sure the graphics do not exceed the margin)
\makeatletter
\def\maxwidth{ %
  \ifdim\Gin@nat@width>\linewidth
    \linewidth
  \else
    \Gin@nat@width
  \fi
}
\makeatother

\definecolor{fgcolor}{rgb}{0.345, 0.345, 0.345}
\newcommand{\hlnum}[1]{\textcolor[rgb]{0.686,0.059,0.569}{#1}}%
\newcommand{\hlstr}[1]{\textcolor[rgb]{0.192,0.494,0.8}{#1}}%
\newcommand{\hlcom}[1]{\textcolor[rgb]{0.678,0.584,0.686}{\textit{#1}}}%
\newcommand{\hlopt}[1]{\textcolor[rgb]{0,0,0}{#1}}%
\newcommand{\hlstd}[1]{\textcolor[rgb]{0.345,0.345,0.345}{#1}}%
\newcommand{\hlkwa}[1]{\textcolor[rgb]{0.161,0.373,0.58}{\textbf{#1}}}%
\newcommand{\hlkwb}[1]{\textcolor[rgb]{0.69,0.353,0.396}{#1}}%
\newcommand{\hlkwc}[1]{\textcolor[rgb]{0.333,0.667,0.333}{#1}}%
\newcommand{\hlkwd}[1]{\textcolor[rgb]{0.737,0.353,0.396}{\textbf{#1}}}%
\let\hlipl\hlkwb

\usepackage{framed}
\makeatletter
\newenvironment{kframe}{%
 \def\at@end@of@kframe{}%
 \ifinner\ifhmode%
  \def\at@end@of@kframe{\end{minipage}}%
  \begin{minipage}{\columnwidth}%
 \fi\fi%
 \def\FrameCommand##1{\hskip\@totalleftmargin \hskip-\fboxsep
 \colorbox{shadecolor}{##1}\hskip-\fboxsep
     % There is no \\@totalrightmargin, so:
     \hskip-\linewidth \hskip-\@totalleftmargin \hskip\columnwidth}%
 \MakeFramed {\advance\hsize-\width
   \@totalleftmargin\z@ \linewidth\hsize
   \@setminipage}}%
 {\par\unskip\endMakeFramed%
 \at@end@of@kframe}
\makeatother

\definecolor{shadecolor}{rgb}{.97, .97, .97}
\definecolor{messagecolor}{rgb}{0, 0, 0}
\definecolor{warningcolor}{rgb}{1, 0, 1}
\definecolor{errorcolor}{rgb}{1, 0, 0}
\newenvironment{knitrout}{}{} % an empty environment to be redefined in TeX

\usepackage{alltt}

\usepackage[left=2cm, right=2cm, top=1cm, bottom=2cm]{geometry}

% Font.


% Main packages.
\usepackage[utf8]{inputenc}
\usepackage[T1]{fontenc}
\usepackage{amsmath}
\usepackage{graphicx}
\usepackage{hyperref}
\usepackage{booktabs} 
\usepackage{rotating} 
\usepackage{lmodern}

% Required for Table.


%%

\title{OeNB Industry Lab - Documentation}
\author{Tristan Leiter}
\date{\today}

%==== DOCUMENT START ==========================================================%

\IfFileExists{upquote.sty}{\usepackage{upquote}}{}
\begin{document}

\maketitle

%==== ABSTRACT ================================================================%

% \begin{abstract}
% This report demonstrates the integration of R code and its output within a LaTeX document using Sweave. It covers the basic structure of a report, including a summary, chapters with subchapters, and a bibliography.
% \end{abstract}

%==== Table of content ========================================================%

\tableofcontents
\newpage

%==== Introduction ============================================================%

\chapter{Einleitung}

\section{Einleitung}




%==== Chapter 1: General ======================================================%

\chapter{Exploratory Data Analysis}

\section{Overview}

\subsection{Description of the dataset}

\subsection{Description of the dataset}

\section{Data Splitting with Imbalanced Data}

\subsection{Class Imbalance}

In this credit risk dataset, the target variable, y, is inherently imbalanced. The number of non-defaults (0) significantly outnumbers the number of defaults (1). This is a common and expected characteristic of credit risk data.

This imbalance poses a significant challenge for model development. If we were to use a simple random split to create our training, validation, and test sets, we would face a high risk of creating unrepresentative samples. For example, a small test set could, purely by chance, end up with a much higher or lower percentage of defaults than the original dataset—or, in the worst case, zero defaults.

\subsection{The Solution: Stratified Sampling}

To prevent this, we employ stratified sampling. This is a technique that ensures the original class distribution of the target variable is preserved in each of the new data splits.

Here is the reasoning for its use:

Guarantees Representation: Stratification forces the splits to maintain the original ratio of defaults to non-defaults. If 0.0865\% of the original dataset are defaults, the training set, validation set, and test set will all contain approximately 0.0865\% defaults.

Enables Reliable Evaluation: When the test set is representative, the performance metrics we calculate (like accuracy, precision, recall, and F1-score) are meaningful. Evaluating a model on a test set with a skewed default rate would give us a misleading and over-optimistic (or pessimistic) score.

Promotes Model Generalization: By training the model on a set that accurately reflects the real-world data distribution, we help it learn the patterns of both the majority (non-default) and minority (default) classes, leading to a more robust and generalizable model.

In summary, using stratified sampling on the y variable is a critical step to ensure our model is trained and evaluated on a reliable, representative foundation.


%==== Chapter 2: ==============================================================%

\chapter{Generalized linear models (GLM)}

\section{Overview}


%==== Anhang ==================================================================%

\appendix 

\chapter{Overview}

%==== END =====================================================================%

\end{document}
